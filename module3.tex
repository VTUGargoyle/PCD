\documentclass[11pt]{beamer}

      \usepackage{hyperref}

      \usepackage{color}

      \usepackage{amsmath}

      \usepackage{listings}
      \lstset{numbers=none,language=[ISO]C++,tabsize=4,
  frame=single,
  basicstyle=\small,
  showspaces=false,showstringspaces=false,
  showtabs=false,
  keywordstyle=\color{blue}\bfseries,
  commentstyle=\color{red},
  }

      \usepackage{verbatim}

\usepackage{fixltx2e}
\usepackage{graphicx}
\usepackage{longtable}
\usepackage{float}
\usepackage{wrapfig}
\usepackage{soul}
\usepackage{textcomp}
\usepackage{marvosym}
\usepackage{wasysym}
\usepackage{latexsym}
\usepackage{amssymb}
\usepackage{hyperref}
\tolerance=1000
\usepackage{minted}
\providecommand{\alert}[1]{\textbf{#1}}

\title{module3}
\author{gar}
\date{}
\hypersetup{
  pdfkeywords={},
  pdfsubject={},
  pdfcreator={Emacs Org-mode version 7.9.3f}}

\begin{document}

\maketitle

\begin{frame}
\frametitle{Outline}
\setcounter{tocdepth}{3}
\tableofcontents
\end{frame}
\section{Functions, Arrays and Strings}
\label{sec-1}
\begin{frame}[fragile]\frametitle{Arrays}
\label{sec-1-1}

\begin{itemize}
\item Collection of elements of same data type
\item When declaring, square brackets are used to indicate that data is an array
\begin{itemize}
\item E.g. If marks of 10 students are to be stored

\begin{minted}[]{C}
int marks[10];
\end{minted}
\item To get the marks of first student

\begin{minted}[]{C}
printf("%d\n", a[0]);
\end{minted}
\item To set the marks of the 10$^{\mathrm{th}}$ student as 65

\begin{minted}[]{C}
a[9] = 65;
\end{minted}
\item We may also declare an array of roll numbers, to know who the n$^{\mathrm{th}}$ student is

\begin{minted}[]{C}
int rollno[10];
\end{minted}
\item It's also possible to logically group the \verb~rollno~ and \verb~marks~ as a structure, which is a topic of module 4
\end{itemize}
\end{itemize}
\end{frame}
\begin{frame}[fragile]\frametitle{Concept of an Array}
\label{sec-1-2}

\begin{itemize}
\item Each number 0, 1, 2 \ldots{} 9 is called \textbf{subscript} or \textbf{index}
\item Square brackets are used for indexing
\item Indexing starts from 0 and goes up to \emph{n}-1, where \emph{n} is the maximum size reserved for an array
\item Initializing an array 

\begin{minted}[]{C}
int a[5] = {-2,-1,0,1,2};
\end{minted}
\begin{itemize}
\item Can also be set individually after declaring

\begin{minted}[]{C}
int a[5];
a[0] = -2;
a[1] = -1;
\end{minted}
\end{itemize}

\item If only one element is Initialized during declaration, all others will be set to zeroes

\begin{minted}[]{C}
int a[5] = {-2,-1};
\end{minted}
  is same as

\begin{minted}[]{C}
int a[5] = {-2,-1,0,0,0};
\end{minted}
\end{itemize}
\end{frame}
\begin{frame}[fragile]\frametitle{Programming Example using Array}
\label{sec-1-3}

\begin{itemize}
\item Read 5 floating point values to an array \verb~a~ and then compute the sum
\end{itemize}

\begin{minted}[]{C}
main()
{
  float a[5], sum = 0;
  int i;
  for (i=0; i<5; i++)
    scanf("%f", &a[i]);
  for (i=0; i<5; i++)
    sum += a[i];
}
\end{minted}

\begin{itemize}
\item Extend it to find
\begin{enumerate}
\item Mean = $\overline{a}$ = $\dfrac{\sum_{i=0}^4 a_i}{5}$
\item Variance = $\sigma$$^2$ =  $\dfrac{\sum_{i=0}^4\left(a_i - \overline{a}\right)^2}{5}$
\end{enumerate}
\end{itemize}
\end{frame}
\begin{frame}[fragile]\frametitle{Array of \emph{char}}
\label{sec-1-4}

\begin{itemize}
\item Array of characters is also called a string
\item Each element can store the ASCII code of the character
\item Every string must end with a null character, which is \verb~\0~
\begin{itemize}
\item When printing the string, all the characters till \verb~\0~ will be printed
\end{itemize}
\item E.g.

\begin{minted}[]{C}
char quote1[20] = "Hello";
char quote2[20] = {'H','e','l','l','o','\0'};
char quote3[20];
quote3[0] = 'H';
quote3[1] = 'e';
quote3[2] = 'l';
quote3[3] = 'l';
quote3[4] = 'o';
quote3[5] = '\0';
\end{minted}
\item \verb~quote1~ and \verb~quote2~ mean exactly the same. But in \verb~quote3~, we need to specify the null character explicitly
\end{itemize}
\end{frame}
\begin{frame}[fragile]\frametitle{Printing and Reading Strings}
\label{sec-1-5}

\begin{itemize}
\item Use the conversion specifier \verb~%s~ when reading and printing

\begin{minted}[]{C}
printf("%s", quote1);
\end{minted}
\item \verb~&~ operator is not required in this case, since specifying the array name implies the address of the string. 

\begin{minted}[]{C}
scanf("%s", quote2);
\end{minted}
\end{itemize}
\end{frame}
\begin{frame}[fragile]\frametitle{Function}
\label{sec-1-6}

\begin{itemize}
\item Provides a convenient way to encapsulate a computation
\item Can later be used without worrying about its implementation
\begin{itemize}
\item \verb~printf()~, \verb~sqrt()~ etc.
\end{itemize}
\item E.g. Define a function to calculate power ($x^y$)

\begin{minted}[]{C}
int power(int x, int y)
{
  int i, p = 1;
  for (i = 1; i <= y; i++)
    p = p*x;
  return p;
}
\end{minted}
\end{itemize}
\end{frame}
\begin{frame}[fragile]\frametitle{Syntax of function definition}
\label{sec-1-7}

\begin{verbatim}
return-type function-name(parmeter declarations, if any)
{
  declarations 
  statements
}
\end{verbatim}

\begin{itemize}
\item Function definitions may appear
\begin{itemize}
\item in any order
\item in one source file or several
\end{itemize}
\end{itemize}
\end{frame}

\end{document}
