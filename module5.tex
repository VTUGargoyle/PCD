\documentclass[11pt]{beamer}

      \usepackage{hyperref}

      \usepackage{color}

      \usepackage{amsmath}

      \usepackage{listings}
      \lstset{numbers=none,language=[ISO]C++,tabsize=4,
  frame=single,
  basicstyle=\small,
  showspaces=false,showstringspaces=false,
  showtabs=false,
  keywordstyle=\color{blue}\bfseries,
  commentstyle=\color{red},
  }

      \usepackage{verbatim}

\usepackage{fixltx2e}
\usepackage{graphicx}
\usepackage{longtable}
\usepackage{float}
\usepackage{wrapfig}
\usepackage{soul}
\usepackage{textcomp}
\usepackage{marvosym}
\usepackage{wasysym}
\usepackage{latexsym}
\usepackage{amssymb}
\usepackage{hyperref}
\tolerance=1000
\usepackage{minted}
\providecommand{\alert}[1]{\textbf{#1}}

\title{module5}
\author{gar}
\date{}
\hypersetup{
  pdfkeywords={},
  pdfsubject={},
  pdfcreator={Emacs Org-mode version 7.9.3f}}

\begin{document}

\maketitle

\begin{frame}
\frametitle{Outline}
\setcounter{tocdepth}{3}
\tableofcontents
\end{frame}
\section{Pointers and Preprocessors}
\label{sec-1}
\begin{frame}[fragile]\frametitle{Pointers}
\label{sec-1-1}

\begin{itemize}
\item Pointer is a variable that contains the address of a variable
\item The memory addresses are numbered consecutively
\item A \verb~char~ variable takes one byte of memory, and can be located anywhere in memory
\item A \verb~short int~ is stored in a pair of adjacent bytes of memory
\item An \verb~int~ is stored in four adjacent bytes of memory
\item In 32-bit addressable systems, \verb~long~ also takes four bytes
\item In 64-bit addressable systems, \verb~long~ takes eight bytes
\item This piece of information is critical when dealing with pointers
\end{itemize}
\end{frame}
\begin{frame}[fragile]\frametitle{Syntax to use Pointers}
\label{sec-1-2}

\begin{itemize}
\item \begin{minted}[]{C}
char ch = 'a';
char *pc = &ch;
printf("%c, %u, %c\n", ch, pc, *pc)
\end{minted}
\item The above set of instructions print `a' (the value in \verb~ch~), the address of \verb~ch~, and again `a'
\item Unary \verb~&~ operator gives the address of a variable
\begin{itemize}
\item but binary \verb~&~ performs \verb~bitwise-and~ of two variables
\end{itemize}
\item Unary \verb~*~ operator is called the dereferencing operator, which when applied to a pointer, it accesses the object the pointer points to
\end{itemize}
\end{frame}
\begin{frame}[fragile]\frametitle{Examples on Pointers}
\label{sec-1-3}

\begin{itemize}
\item \begin{minted}[]{C}
char c = 'C';
int i = 256;
char *pc = &c;
char *pi = &i;
printf("%d\n", *pi);
\end{minted}
\item In the example, \verb~char *pc~ means that \verb~pc~ points to a character
\item But \verb~pi~ also has been declared as a pointer to a character
\item The compilation is possible, but a warning like ``incompatible pointer type'' will be issued
\item Whatever variable a pointer points to, the size of all the pointers will be the same
\begin{itemize}
\item because they contain address, which will be 32 bits or 64 bits depending on the compiler and the OS
\end{itemize}
\item The incompatible pointer will cause problems when dereferencing
\item In the example, 0 is printed instead of 256
\begin{itemize}
\item because $256=2^8$, which means $100000000$ in binary
\item and only the last 8 bits are accessed since we have declared it as a pointer to a \verb~char~
\end{itemize}
\end{itemize}
\end{frame}
\begin{frame}[fragile]\frametitle{Pointers and Function Arguments}
\label{sec-1-4}

\begin{itemize}
\item Pointers can be used to access elements in another function
\item Consider a function \verb~inc5~ supposed to increment a variable by 5 

\begin{minted}[]{C}
void inc5(int i)
{ 
  i = i+5;
}
\end{minted}
\item If we call \verb~inc5(a)~, variable \verb~a~ will not be incremented since a copy of \verb~a~ is passed (pass by value)
\item Now, modify the function

\begin{minted}[]{C}
void inc5(int *i)
{ 
  *i = *i+5;
}
\end{minted}
\item If we call \verb~inc5(&a)~, variable \verb~a~ will be incremented since address of \verb~a~ is passed (pass by reference)
\end{itemize}
\end{frame}
\begin{frame}[fragile]\frametitle{Pointers and Arrays}
\label{sec-1-5}

\begin{itemize}
\item An array \verb~a~ defined as

\begin{minted}[]{C}
int a[5] = {10, 20, 30, 40, 50};
\end{minted}
  will contain the address of the first element
\item \verb~a[1]~ will access the next element
\begin{itemize}
\item which is same as \verb~*(a+1)~
\end{itemize}
\item We may define a pointer to array 

\begin{minted}[]{C}
int *pa = &a[0]; /* points to first element of a */
int *pa2 = a;    /* also means the same */
\end{minted}
\item The only difference is that the pointer is a variable, whereas the array name is not
\begin{itemize}
\item So, \verb~pa++;~ will point to the next element, but \verb~a++;~ is an error since \verb~a~ cannot change
\end{itemize}
\end{itemize}
\end{frame}
\begin{frame}[fragile]\frametitle{Address Arithmetic}
\label{sec-1-6}

\begin{itemize}
\item If \verb~pa~ is a pointer to an array, then \verb~pa++;~ increments the value of \verb~pa~ such that it points to the next element in the array
\item A small program written and run in a computer will verify it

\begin{minted}[]{C}
int main ()
{
  int a[5] = {1,2,3,4,5};
  char c[] = "tring";
  int *pa = a;
  char *pc = c;
  printf("%u %u\n", pa,pa+1); /* 3899203328 3899203332 */
  printf("%u %u\n", pc,pc+1); /* 3899203360 3899203361 */
  return 0;
}
\end{minted}
\end{itemize}
\end{frame}
\begin{frame}[fragile]\frametitle{Character Pointers and Functions}
\label{sec-1-7}

\begin{itemize}
\item We may also have a pointer to a string

\begin{minted}[]{C}
char a[] = "a string";
char *ps = "yet another string";
\end{minted}
\item \verb~a~ is an array name, which contains the starting address of the string
\item The string will be stored somewhere in memory, and its starting address will be assigned to \verb~ps~
\end{itemize}
\end{frame}
\begin{frame}[fragile]\frametitle{Character Pointers and Functions}
\label{sec-1-8}

\begin{itemize}
\item Write a function \verb~strlen~ to compute the length of a string using pointers

\begin{minted}[]{C}
int strlen(char *s)
{
    int n;
    for (n = 0; *s != '\0'; s++)
        n++;
    return n;
}
\end{minted}
\item We may then call the function in multiple ways:

\begin{minted}[]{C}
strlen("hello, strlen"); /* string constant */
strlen(a);               /* char a[]; */
strlen(ps);              /* char *ps */
\end{minted}
\end{itemize}
\end{frame}
\begin{frame}[fragile]\frametitle{Character Pointers and Functions}
\label{sec-1-9}

\begin{itemize}
\item Write a function \verb~strcpy~ to copy a source string to destination string

\begin{minted}[]{C}
void strcpy(char *s, char *t)
{
    while ((*s = *t) != '\0') {
        s++;
        t++;
    }
}
\end{minted}
\item which can be equivalently shortened to

\begin{minted}[]{C}
void strcpy(char *s, char *t)
{
    while ((*s++ = *t++) != '\0') ;
}
\end{minted}
\end{itemize}
\end{frame}
\begin{frame}[fragile]\frametitle{Pointer Arrays}
\label{sec-1-10}

\begin{itemize}
\item Since pointers are variables, they can also be stored in arrays
\item One of the useful applications of such an array is to sort names

\begin{minted}[]{C}
char *ps[] = {"b1", "a12", "c3"};
\end{minted}
\item The three strings are in different memory locations, and the pointer array holds addresses in the order of the strings
\item Now, to sort names, instead of swapping the complete string character by character, we only swap the first two addresses in the pointer array
\end{itemize}
\end{frame}
\begin{frame}[fragile]\frametitle{Dynamic Memory Allocation}
\label{sec-1-11}

\begin{itemize}
\item Sometimes, the size of the input data will be unknown in advance
\item Allocating the maximum possible size may waste a lot of space
\begin{itemize}
\item E.g. If we are going to get an array of up to 1000 integers, then declaration like \verb~int num[1000];~ can be made,
\item But, it wastes space if we get lesser number of elements
\end{itemize}
\item So, dynamic memory allocation is used, which will allocate space for the variables when it's required
\end{itemize}
\end{frame}
\begin{frame}[fragile]\frametitle{Dynamic Memory Allocation: \verb~malloc~}
\label{sec-1-12}

\begin{itemize}
\item \verb~malloc~ is one function which can allocate space as needed
\item Its prototype is described in \verb~stdlib.h~
\item Its usage is:

\begin{minted}[]{C}
ptr = (type) malloc(size);
\end{minted}
\begin{itemize}
\item \verb~malloc~ returns a pointer to the memory allocated
\item It's of type \verb~void *~, called a \emph{generic pointer} and must be explicitly typecast to the appropriate data type
\end{itemize}
\end{itemize}
\end{frame}
\begin{frame}[fragile]\frametitle{Dynamic Memory Allocation: \verb~malloc~ example}
\label{sec-1-13}


\begin{minted}[]{C}
struct emp {
    char name[20];
    int empnum;
    double salary;
};
struct emp *worker;
worker = (struct emp *) malloc(sizeof(struct emp));

worker -> empnum = 1;
\end{minted}
\begin{itemize}
\item \verb~sizeof~ returns the required number of bytes for the structure
\item \verb~malloc~ reserves the space and returns the address of the space, which is converted to the structure data type
\end{itemize}
\end{frame}
\begin{frame}[fragile]\frametitle{Array of Pointers}
\label{sec-1-14}

\begin{itemize}
\item If more than one employee is information is to be stored, an array of pointers can be declared

\begin{minted}[]{C}
struct emp {
    char name[20];
    int empnum;
    double salary;
};
struct emp *worker[20];
worker[3] = (struct emp *) malloc(sizeof(struct emp));

worker[3] -> empnum = 4;
\end{minted}
\item Whenever a new employee is hired, the index value is incremented, which then points to the new employee
\end{itemize}
\end{frame}
\begin{frame}[fragile]\frametitle{Freeing Memory}
\label{sec-1-15}

\begin{itemize}
\item After using the allocated memory, we need to \verb~free~ it so that it can be re-used
\item Its general form is 

\begin{minted}[]{C}
free(ptr);
\end{minted}
\item \verb~ptr~ must be pointing to some memory address
\end{itemize}
\end{frame}
\begin{frame}[fragile]\frametitle{Preprocessor Directives}
\label{sec-1-16}

\begin{itemize}
\item Just before compiling a program, it involves a preprocessing stage
\item Preprocessor modifies the source code before it is handed over to the compiler
\item Such modifications are indicated by preprocessor directives, which are indicated by \verb~#~ symbol
\item It provides the ability for the inclusion of header files, macro expansions, conditional compilation, etc.
\end{itemize}
\end{frame}
\begin{frame}[fragile]\frametitle{\#define directive}
\label{sec-1-17}

\begin{itemize}
\item \#define directive is used to substitute some text in the source code
\item It's also called a macro
\item The syntax is 

\begin{minted}[]{C}
#define identifier <substitute text>
\end{minted}
\item Example 

\begin{minted}[]{C}
#define PI 3.14159265359
\end{minted}
\begin{itemize}
\item Replaces every occurance of \verb~PI~ with the defined value
\end{itemize}
\end{itemize}
\end{frame}
\begin{frame}[fragile]\frametitle{(Extra) Checking the Preprocessor Output}
\label{sec-1-18}

\begin{itemize}
\item \verb~gcc~ provides with an option \verb~-E~, which enables to see the output of the preprocessing stage
\item E.g. 

\begin{minted}[]{C}
/* store program as pi.c */
#define PI 3.14159265359

int main()
{
  printf("%f\n", PI/2);
  return 0;
}
\end{minted}
\begin{itemize}
\item \verb~gcc -E pi.c~
\end{itemize}
\item Check the output and observe the changes
\end{itemize}
\end{frame}
\begin{frame}[fragile]\frametitle{Macros with Arguments}
\label{sec-1-19}

\begin{itemize}
\item Macros can also receive parameters
\item E.g.

\begin{minted}[]{C}
#define DOUBLE(a) (a)*2

int main()
{
    printf("%d", DOUBLE(5+3));
    return 0;
}
\end{minted}
\begin{itemize}
\item \verb~DOUBLE(5+3)~ is substituted with (5+3)*2 after preprocessing
\end{itemize}
\end{itemize}
\end{frame}
\begin{frame}[fragile]\frametitle{Undefining a Macro}
\label{sec-1-20}

\begin{itemize}
\item A macro defined with \#define can be undefined using \#undef directive
\item That macro can then not be used after undefining
\item Useful when we are trying to redifine a macro to a new value
\item E.g. to assign an approximate value of pi to a macro \verb~M_PI~ defined in \verb~math.h~

\begin{minted}[]{C}
#include <math.h>
#include <stdio.h>

#undef M_PI
#define M_PI (22/7.0)
\end{minted}
\end{itemize}
\end{frame}
\begin{frame}[fragile]\frametitle{\#include directive}
\label{sec-1-21}

\begin{itemize}
\item The \#include directive loads the specified file in the program
\item The included file is also compiled with the program
\item Two ways of including

\begin{minted}[]{C}
#include <filename> /* 1 */

#include "filename" /* 2 */
\end{minted}
\item The header files  stored in standard directories will be included if \verb~< >~ is used
\item The header files stored in current and standard directories will be included if `` `` is used
\end{itemize}
\end{frame}
\begin{frame}[fragile]\frametitle{Conditional Compilation}
\label{sec-1-22}

\begin{itemize}
\item The statements are compiled only if some condition is true
\item Syntax

\begin{minted}[]{C}
#ifdef <identifier>
{
statements;
}
#else
{
statements;
}
#endif
\end{minted}
\end{itemize}
\end{frame}
\begin{frame}[fragile]\frametitle{Conditional Compilation}
\label{sec-1-23}

\begin{itemize}
\item E.g.
\end{itemize}
{\scriptsize 

\begin{minted}[]{C}
#include <stdio.h>
#include <math.h>
#define E =
int main()
{
    int a;
    #ifdef E
    {
        a E 1;
    }
    #else
    {
        a = 2;
    }
    #endif
    printf("%d", a);
    return 0;
}
\end{minted}
}
\end{frame}
\begin{frame}[fragile]\frametitle{Data Structures}
\label{sec-1-24}

\begin{itemize}
\item Data Structure is a method of storing data in a computer so that it may be used efficiently
\item Data Structure
\begin{itemize}
\item Primitive
\begin{itemize}
\item \verb~int~, \verb~char~, \verb~float~ etc.
\end{itemize}
\item Non-Primitive
\begin{itemize}
\item Linear (Arrays, stacks, queues, linked-lists)
\item Non-linear (Trees, graphs)
\end{itemize}
\end{itemize}
\end{itemize}
\begin{itemize}
\item The basic data type provided by the programming language is called the primitive data type
\item The data type derived from basic types is called non-primitive data types
\end{itemize}
\end{frame}
\begin{frame}[fragile]\frametitle{Stack}
\label{sec-1-25}

\begin{itemize}
\item Stack is a data structure where the elements are inserted to one end and deleted from the same end
\item The position where the insertion and deletion happens is called the top of the stack
\item Also called Last-In-First-Out (LIFO) data structure
\item Three main stack operations
\begin{itemize}
\item Push
\item Pop
\item Display
\end{itemize}
\end{itemize}
\end{frame}
\begin{frame}[fragile]\frametitle{Queue}
\label{sec-1-26}

\begin{itemize}
\item Stack is a data structure where the elements are inserted to one end and deleted from the other end
\item The end where the elements are inserted is called the rear end
\item The end where the elements are deleted is called the front end
\item Also called First-In-First-Out (FIFO) data structure
\item Operations on queues
\begin{itemize}
\item Insert (Enqueue)
\item Delete (Dequeue)
\item Display
\end{itemize}
\end{itemize}
\end{frame}
\begin{frame}[fragile]\frametitle{Linked List}
\label{sec-1-27}

\begin{itemize}
\item A data strucutre which is a collection of zero or more nodes where each node has some information
\item Node consists two fields
\begin{itemize}
\item \verb~info~: which holds some information
\item \verb~link~: contains the address of the next node
\end{itemize}
\item Types of linked list
\begin{itemize}
\item Singly linked lists
\begin{itemize}
\item Last node's \verb~link~ field will be \verb~NULL~
\end{itemize}
\item Doubly linked lists
\begin{itemize}
\item Contains two \verb~link~ fields, for right and left nodes
\end{itemize}
\item Circular singly linked lists
\begin{itemize}
\item Last node's \verb~link~ field will contain the address of the first node
\end{itemize}
\item Circular doubly linked lists
\end{itemize}
\end{itemize}
\end{frame}
\begin{frame}[fragile]\frametitle{Singly Linked Lists}
\label{sec-1-28}

\begin{itemize}
\item Operations on singly linked lists
\begin{itemize}
\item Inserting a node
\item Deleting a node
\item Search in a list
\item Display the contents
\end{itemize}
\end{itemize}
\end{frame}
\begin{frame}[fragile]\frametitle{Trees}
\label{sec-1-29}

\begin{itemize}
\item Non-empty set of items where one element is called a root, and the remaining items are divided into $n\ge 0$ disjoint subsets,
  each of which can be a tree
\item Every item is called a node
\item Every node can have zero or more branches (subtrees)
\end{itemize}
\end{frame}

\end{document}
