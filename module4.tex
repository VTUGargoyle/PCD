\documentclass[11pt]{beamer}

      \usepackage{hyperref}

      \usepackage{color}

      \usepackage{amsmath}

      \usepackage{listings}
      \lstset{numbers=none,language=[ISO]C++,tabsize=4,
  frame=single,
  basicstyle=\small,
  showspaces=false,showstringspaces=false,
  showtabs=false,
  keywordstyle=\color{blue}\bfseries,
  commentstyle=\color{red},
  }

      \usepackage{verbatim}

\usepackage{fixltx2e}
\usepackage{graphicx}
\usepackage{longtable}
\usepackage{float}
\usepackage{wrapfig}
\usepackage{soul}
\usepackage{textcomp}
\usepackage{marvosym}
\usepackage{wasysym}
\usepackage{latexsym}
\usepackage{amssymb}
\usepackage{hyperref}
\tolerance=1000
\usepackage{minted}
\providecommand{\alert}[1]{\textbf{#1}}

\title{module4}
\author{gar}
\date{}
\hypersetup{
  pdfkeywords={},
  pdfsubject={},
  pdfcreator={Emacs Org-mode version 7.9.3f}}

\begin{document}

\maketitle

\begin{frame}
\frametitle{Outline}
\setcounter{tocdepth}{3}
\tableofcontents
\end{frame}
\section{Structures and File Management}
\label{sec-1}
\begin{frame}[fragile]\frametitle{Structure}
\label{sec-1-1}

\begin{itemize}
\item Used to group data values which are heterogeneous, and the data are related an entity
\begin{itemize}
\item Array groups homogeneous data
\end{itemize}
\item Structure is declared using \verb~struct~ data type
\item E.g. To store a student name, roll number and branch, a structure can be defined as:

\begin{minted}[]{C}
struct student {
  char name[20];
  int rollno;
  char branch[20];
};
\end{minted}
\item \verb~name~, \verb~rollno~ and \verb~branch~ are called the \emph{members} of the structure, which can be of any data type
\end{itemize}
\end{frame}
\begin{frame}[fragile]\frametitle{General Form of \verb~struct~ definition}
\label{sec-1-2}


\begin{minted}[]{C}
struct struct_tag {
  type_1 var_1;
  type_2 var_2;
  .
  .
  type_n var_n;
} structvar1, structvar2;
\end{minted}
\end{frame}
\begin{frame}[fragile]\frametitle{Accessing members of a structure}
\label{sec-1-3}

\begin{itemize}
\item A dot operator is used to access the members
\item E.g. consider a struct \verb~point~ used to store x and y coordinates of a point:

\begin{minted}[]{C}
struct point {
  int x;
  int y;
};
\end{minted}
\item Declare and initialize a variable \verb~p~ 

\begin{minted}[]{C}
struct point p = {1,2};
\end{minted}
\item To print value x of p

\begin{minted}[]{C}
printf("%d\n", p.x);
\end{minted}
\item To set a new value of x

\begin{minted}[]{C}
p.x = 5;
\end{minted}
\end{itemize}
\end{frame}
\begin{frame}[fragile]\frametitle{Questions}
\label{sec-1-4}

\begin{enumerate}
\item Define a structure \verb~book~, which must store the title of the book, author name, year of publication and the publisher's name
\item Define a structure \verb~rectangle~, which stores two \verb~point~ structures (nested structure), to indicate the bottom left and top right points of a rectangle
\end{enumerate}
\end{frame}
\begin{frame}[fragile]\frametitle{Nested Structure}
\label{sec-1-5}

\begin{itemize}
\item Defining a \verb~rectangle~ structure (initializing and accessing the values)
\end{itemize}


\begin{minted}[]{C}
struct rectangle {
    struct point p1; /* bottom-left */
    struct point p2; /* top-right */
};

int main()
{
    struct rectangle r = {{1,2},{5,5}};
    printf("%d %d\n", r.p1.x, r.p1.y);
    printf("%d %d\n", r.p2.x, r.p2.y);
}
\end{minted}
\end{frame}
\begin{frame}[fragile]\frametitle{Array of Structures}
\label{sec-1-6}

\begin{itemize}
\item Array of struct data type is also possible:
\begin{itemize}
\item E.g. 

\begin{minted}[]{C}
struct point p[10];
\end{minted}
   declares an array of 10 points, and reading of the data can be done using dot operator

\begin{minted}[]{C}
for (i=0; i<10; i++)
  scanf("%d%d", &p[i].x, &p[i].y);
\end{minted}
\end{itemize}
\end{itemize}
\end{frame}
\begin{frame}[fragile]\frametitle{Structure and Function}
\label{sec-1-7}

\begin{itemize}
\item When passing a structure to a function, it's passed by value instead of reference
\item Changing values in the called function will not change the structure variables in the calling function
\item E.g.

\begin{minted}[]{C}
void set(struct point p, int x, int y)
{
  p.x = x;
  p.y = y;
}
main()
{
  struct point a;
  set(a, x, y);
}
\end{minted}
  will not set the values of \verb~a~, since a copy of the structure is sent instead of the address
\end{itemize}
\end{frame}
\begin{frame}[fragile]\frametitle{Structure and Function}
\label{sec-1-8}

\begin{itemize}
\item To set the values of a structure, pass by reference and use \verb~->~ operator

\begin{minted}[]{C}
void set(struct point *p, int x, int y)
{
  p->x = x; /* (*p).x = x; also is fine */
  p->y = y;
}
main()
{
  struct point a;
  set(&a, x, y);
}
\end{minted}
\end{itemize}
\end{frame}
\begin{frame}[fragile]\frametitle{Sending Array of Structures as Parameter}
\label{sec-1-9}


\begin{minted}[]{C}
struct candidate {
    char name[50];
    int age;
};
void read_data(struct candidate c[], int n)
{
    int i;
    for (i=0; i<n; i++) {
        scanf("%s", c[i].name);
        scanf("%d", &c[i].age);
    }
}
int main ()
{
    struct candidate cand[10];
    read_data(cand, 2);
    return 0;
}
\end{minted}
\end{frame}
\begin{frame}[fragile]\frametitle{Type Definitions}
\label{sec-1-10}

\begin{itemize}
\item C provides a keyword \verb~typedef~ for creating new data type names
\item E.g. If we are going to use \verb~short int~ to store dimensions of rectangles, \verb~typedef~ can be used

\begin{minted}[]{C}
typedef short int Length;
\end{minted}
\item Then we may use the data type \verb~Length~ instead of \verb~short int~

\begin{minted}[]{C}
Length h, w; /* represents height and width */
\end{minted}
\item It depicts the intentions of the variables more clearly
\end{itemize}
\end{frame}
\begin{frame}[fragile]\frametitle{Type Definitions}
\label{sec-1-11}

\begin{itemize}
\item As another example, a structure can be typedef in two ways

\begin{minted}[]{C}
struct point {
  int x;
  int y;
};
typedef struct point Point;
\end{minted}
\item or 

\begin{minted}[]{C}
typedef struct point {
  int x;
  int y;
} Point; /* In this case, point tag is optional */
\end{minted}
\item We may use the new data type to declare point structures

\begin{minted}[]{C}
main()
{
  Point p = {1,5};
  ...
}
\end{minted}
\end{itemize}
\end{frame}
\begin{frame}[fragile]\frametitle{File Management}
\label{sec-1-12}

\begin{itemize}
\item A file is a collection of data stored on disk
\item C provides with several functions to work with files
\item A file must be opened by using the function \verb~fopen~, which returns a pointer to the opened file
\begin{itemize}
\item it's called a \emph{file pointer}
\end{itemize}
\end{itemize}
\end{frame}
\begin{frame}[fragile]\frametitle{Opening a File}
\label{sec-1-13}

\begin{itemize}
\item General form of a call to \verb~fopen~ is

\begin{minted}[]{C}
filepointer = fopen(filename, mode);
\end{minted}
\item Example usage, to open a file \verb~names.txt~ in \verb~read-only~ mode:

\begin{minted}[]{C}
/* Opened in read-mode, other modes: "w", "a" */
FILE *fp = fopen("names.txt", "r"); 
if (fp == NULL)
  printf("Error opening the file\n");
\end{minted}
\item If the file is not found, or necessary permissions are not there for the program, a NULL pointer will be returned
\item Otherwise, a pointer to that file will be returned
\item ``w'' opens a file to write. If the file already exists, it will be overwritten, otherwise it will be created
\item ``a'' opens a file to append. If the file already exists, it will append output to end of the file, otherwise it will be created
\end{itemize}
\end{frame}
\begin{frame}[fragile]\frametitle{Output using fprintf}
\label{sec-1-14}

\begin{itemize}
\item General form:

\begin{minted}[]{C}
fprintf(destfile, "format string", list of variables);
\end{minted}
\item Sending output to a file

\begin{minted}[]{C}
FILE *fo;
int num = 119;

fo = fopen("nums.txt", "w"); 
if (fo != NULL)
  fprintf(fo, "%d\n", num);
\end{minted}
\end{itemize}
\end{frame}
\begin{frame}[fragile]\frametitle{Input using fscanf}
\label{sec-1-15}

\begin{itemize}
\item General form

\begin{minted}[]{C}
fscanf(sourcefile, "format string", list of variables);
\end{minted}
\item Receiving input from a file

\begin{minted}[]{C}
FILE *fi;
int num;

fi = fopen("nums.txt", "r");
if (fi != NULL)
  fscanf(fi, "%d", &num);
\end{minted}
\item This reads a number from the file \verb~nums.txt~ and stores in the variable num
\end{itemize}
\end{frame}
\begin{frame}[fragile]\frametitle{\verb~stdin~ and \verb~stdout~}
\label{sec-1-16}

\begin{itemize}
\item \verb~stdin~ is called an input stream (standard input) and \verb~stdout~ is the output stream (standard output)
\item The two statements below mean the same

\begin{minted}[]{C}
fscanf(stdin, "%d", &num);
scanf("%d", &num);
\end{minted}
\item Similarly, the two statements below mean the same

\begin{minted}[]{C}
fprintf(stdout, "%d", num);
printf("%d", num);
\end{minted}
\item Every program is provided with these pointers by default
\item Another stream is available, which is called \verb~stderr~, mainly used for redirecting error and diagnostic messages
\end{itemize}
\end{frame}
\begin{frame}[fragile]\frametitle{Closing files}
\label{sec-1-17}

\begin{itemize}
\item After a file is used, it must be closed
\item It's general form is

\begin{minted}[]{C}
fclose(filepointer);
\end{minted}
\end{itemize}
\end{frame}
\begin{frame}[fragile]\frametitle{String input from a file}
\label{sec-1-18}


\begin{minted}[]{C}
strptr = fgets(inputarea, n, source);
\end{minted}
\begin{itemize}
\item Example

\begin{minted}[]{C}
char inarea[15];
char *instring;

instring = fgets(inarea, 15, stdin);
\end{minted}
\item fgets reads at most 14 characters, then appends one NULL character
\begin{itemize}
\item If the entered string is : ``Hello, world string''
\item The string stored will be

\begin{center}
\begin{tabular}{|c|c|c|c|c|c|c|c|c|c|c|c|c|c|c|}
\hline
 H  &  e  &  l  &  l  &  o  &  ,  &     &  w  &  o  &  r  &  l  &  d  &     &  s  &  \textbackslash 0  \\
\hline
\end{tabular}
\end{center}


\end{itemize}
\end{itemize}
\end{frame}
\begin{frame}[fragile]\frametitle{String output to a file}
\label{sec-1-19}

\begin{itemize}
\item General form

\begin{minted}[]{C}
fputs(string, destfile);
\end{minted}
\item Example: echoing back whatever is typed

\begin{minted}[]{C}
char inarea[20];
char *instring;

instring = fgets(inarea, 20, stdin);
while (instring != NULL) {
  fputs(instring, stdout);
  instring = fgets(inarea, 20, stdin);
}
\end{minted}
\end{itemize}
\end{frame}
\begin{frame}[fragile]\frametitle{File I/O for characters: fgetc}
\label{sec-1-20}

\begin{itemize}
\item General form of \verb~fgetc~

\begin{minted}[]{C}
ch = fgetc(source);
\end{minted}
\item Reads character from a file pointed by source
\item Returns character converted to \verb~int~, or \verb~EOF~ at the end of file
\end{itemize}
\end{frame}
\begin{frame}[fragile]\frametitle{File I/O for characters: fputc}
\label{sec-1-21}

\begin{itemize}
\item General form of \verb~fputc~

\begin{minted}[]{C}
fputc(ch, destfile);
\end{minted}
\item \verb~ch~ is a character, and \verb~destfile~ is a pointer to a file or stdout
\end{itemize}
\end{frame}
\begin{frame}[fragile]\frametitle{Example Programs}
\label{sec-1-22}

\begin{enumerate}
\item Write a program to replace all the letter `a' with letter `A', where the text is read from ``indata.txt'' and the output is stored in ``outdata.txt''
\item Write a program to count the number of full stops and commas in the file named ``howmany.txt'', and display the result in the terminal
\end{enumerate}
\end{frame}
\begin{frame}[fragile]\frametitle{Program 1}
\label{sec-1-23}

{\scriptsize 

\begin{minted}[]{C}
#include <stdio.h>

int main()
{
    FILE *fi = fopen("indata.txt", "r");
    FILE *fo = fopen("outdata.txt", "w");
    int ch;
    if (fo != NULL) {
        ch = fgetc(fi);
        while (ch != EOF) {
            if (ch == 'a') {
                fputc('A', fo);
            } else {
                fputc(ch, fo);
            }
            ch = fgetc(fi);
        }
    }
    fclose(fi);
    fclose(fo);
    return 0;
}
\end{minted}
}
\end{frame}
\begin{frame}[fragile]\frametitle{Program 2}
\label{sec-1-24}

{\scriptsize 

\begin{minted}[]{C}
#include <stdio.h>

int main()
{
    FILE *fi;
    int comma = 0, dots = 0, ch;
    fi = fopen("howmany.txt", "r");
    if (fi != NULL) {
        ch = fgetc(fi);
        while (ch != EOF) {
            if (ch == '.')
                dots++;
            if (ch == ',')
                comma++;
            ch = fgetc(fi);
        }
    }
    fclose(fi);
    printf("commas: %d, full stops: %d\n", comma, dots);
    return 0;
}
\end{minted}
}
\end{frame}

\end{document}
